\documentclass{article}

\usepackage{bytefield}
\usepackage{mathtools}
\usepackage{float}
\usepackage{url}

\def\r#1{\texttt{r#1}}
\def\pc{\texttt{pc}}

\def\band{\mathrel{\&}}
\def\bxor{\mathrel{\hat{}}}
\def\bor{\mathrel{|}}
\def\bneg{\mathord{\sim}}

\begin{document}

\section{ARM Introduction}

In this assignment, you'll be writing some short programs directly in machine
code. You won't quite write them directly as 1's and 0's---you'll actually type
in hexadecimal characters---but you'll directly use the encoding of ARM
instructions in binary. This section lays out the instructions you'll need to
use for this assignment in detail.

For now, it will suffice to know a few basic formats for data manipulation
instructions. Here is the format for manipulating data strictly between
registers:

\begin{figure}[H]
  \centering
  \begin{bytefield}{32}
    \bitheader[endianness=big]{0-31} \\
    \bitboxes*{1}{1110000} &
    \bitbox{4}{opcode} &
    \bitbox{1}{0} &
    \bitbox{4}{$R_n$} &
    \bitbox{4}{$R_d$} &
    \bitboxes*{1}{00000000} &
    \bitbox{4}{$R_m$}
  \end{bytefield}
\end{figure}

Generally, $R_n$ and $R_m$ are the operands of this operation; opcode
selects which operator to apply to them, and the result is stored in
$R_d$.  Some math-related opcodes useful for this assignment are:

\begin{figure}[H]
  \centering
  \begin{tabular}{lll}
    opcode & mnemonic & expression \\
    \hline \\
    0000 & AND & $R_d \coloneqq R_n \band R_m$ \\
    0001 & EOR & $R_d \coloneqq R_n \bxor R_m$ \\
    0010 & SUB & $R_d \coloneqq R_n - R_m$ \\
    0011 & RSB & $R_d \coloneqq R_m - R_n$ \\
    0100 & ADD & $R_d \coloneqq R_n + R_m$ \\
    1100 & ORR & $R_d \coloneqq R_n \bor R_m$ \\
    1110 & BIC & $R_d \coloneqq R_m \band \bneg R_m$ \\
  \end{tabular}
\end{figure}

Another form of data manipulation instructions lets us combine values in
registers with constants:


\begin{figure}[H]
  \centering
  \begin{bytefield}{32}
    \bitheader[endianness=big]{0-31} \\
    \bitboxes*{1}{1110001} &
    \bitbox{4}{opcode} &
    \bitbox{1}{0} &
    \bitbox{4}{$R_n$} &
    \bitbox{4}{$R_d$} &
    \bitboxes*{1}{0000} &
    \bitbox{8}{Imm}
  \end{bytefield}
\end{figure}

The key difference is the {\tt 1} instead of a {\tt 0} in bit 25, which makes
the instruction operate in immediate mode. Then, Imm is a binary value that is
used in place of $R_m$ above. So, for example, if we used {\tt 00001111} as
bits 0-7 of of an addition instruction, it would add 15 to the value of the
$R_n$ register and store it in $R_d$.


There is another format of instructions which is used for multiply-like
instructions:

\begin{figure}[H]
  \centering
  \begin{bytefield}{32}
    \bitheader[endianness=big]{0-31} \\
    \bitboxes*{1}{11100000} &
    \bitbox{3}{op} &
    \bitbox{1}{0}
    \bitbox{4}{$R_d$} &
    \bitboxes*{1}{0000} &
    \bitbox{4}{$R_n$} &
    \bitboxes*{1}{1001} &
    \bitbox{4}{$R_m$}
  \end{bytefield}
\end{figure}

The multiply operation's opcode is {\tt 000}, so replacing \textit{op} above
with {\tt 000} indicates a multiply. You only need this version of multiply for
this assignment, you can see the others in tabular form in appendix B of the
Harris book if you're interested in seeing more.

\section{Support Code and Example}

We've provided support code so that you can simply write machine code
instructions as hexadecimal, create a binary file from them, and then execute
it with certain starting conditions. Let's consider the following problem:

\begin{quote}
Assume that \r{1} contains $x$ and \r{2} contains $y$. Write machine code
instructions that result in the value $x + y$ being stored in \r{0}.
\end{quote}

Our task is to create a binary file containing the (in this case single)
instruction that adds the values and stores the result in \r{0}. We can use the
specification of add above to get started; we want to take the pattern for data
processing instructions, and pick the opcode for ADD (which is {\tt 0 1 0 0})
and the appropriate registers. Note that the registers are indicated by their
unsigned binary representation in 4 bits, so \r{0} is {\tt 0 0 0 0}, \r{1} is
{\tt 0 0 0 1}, and so on. Here's the first step

\begin{figure}[H]
  \centering
  \begin{bytefield}{32}
    \bitheader[endianness=big]{0-31} \\
    \bitboxes*{1}{1110000} &
    \bitbox{4}{opcode} &
    \bitbox{1}{0} &
    \bitbox{4}{$R_n$} &
    \bitbox{4}{$R_d$} &
    \bitboxes*{1}{00000000} &
    \bitbox{4}{$R_m$}
  \end{bytefield}

  \vspace{2em}

  \begin{bytefield}{32}
    \bitheader[endianness=big]{0-31} \\
    \bitboxes*{1}{1110000} &
    \bitboxes*{1}{0100} &
    \bitbox{1}{0} &
    \bitboxes*{1}{0010} &
    \bitboxes*{1}{0000} &
    \bitboxes*{1}{00000000} &
    \bitboxes*{1}{0001}
  \end{bytefield}
\end{figure}

(Here, we picked that the $n$-labelled source register would hold \r{2} ($y$),
and the $m$-labelled source register would hold \r{1} ($x$). Since addition
commutes, we could have picked either order.)

Now we need to take that instruction and turn it into a binary file. Note that
a file containing \emph{text} 1's and 0's is not a binary file (a good resource
on this is
\url{https://www.cs.umd.edu/class/sum2003/cmsc311/Notes/BitOp/asciiBin.html}).
We don't want a file containing a sequence of bytes representing ASCII; we want
a file containing exactly the four bytes of the instruction above, in the right
format and order.

One of the most straightforward ways to do this is with the {\tt xxd} command.
It is useful for a number of things, where one mode of operation is taking
files containing hexadecimal text and converting them to the corresponding
binary files. We'll show an example of that command in a moment, but first, we
need to think a little bit about the ordering of bits.

All of the instructions we've shown have had the 0th bit all the way to the
right. This is a convention for writing out ARM machine code because it reads
well left-to-right -- first the opcode, then the arguments. However, the actual
order of the bits is the reverse of this. Further, the bytes themselves are
interpreted in what's called ``little-endian'' order, meaning the ``ones''
place is actually all the way to the left. Let's do an example to process this
conversion.

First, we re-organize the breaks to see the instruction in terms of byte
chunks, which will help us think through the hexadecimal encoding.

\begin{figure}[H]
  \centering
  \begin{bytefield}{32}
    \bitheader[endianness=big]{0-31} \\
    \bitboxes*{1}{1110000} &
    \bitboxes*{1}{0100} &
    \bitbox{1}{0} &
    \bitboxes*{1}{0010} &
    \bitboxes*{1}{0000} &
    \bitboxes*{1}{00000000} &
    \bitboxes*{1}{0001}
  \end{bytefield}
\end{figure}

\begin{figure}[H]
  \centering
  \begin{bytefield}{32}
    \bitheader[endianness=big]{0-31} \\
    \bitboxes*{1}{11100000} &
    \bitboxes*{1}{10000010} &
    \bitboxes*{1}{00000000} &
    \bitboxes*{1}{00000001} &
  \end{bytefield}
\end{figure}

Next we reverse the whole thing to help us see the correct left-to-right order
of the whole instruction:

\begin{figure}[H]
  \centering
  \begin{bytefield}{32}
    \bitheader[endianness=little]{0-31} \\
    \bitboxes*{1}{10000000} &
    \bitboxes*{1}{00000000} &
    \bitboxes*{1}{01000001} &
    \bitboxes*{1}{00000111} &
  \end{bytefield}
\end{figure}

Next, we flip each individual byte, because the bytes are interpreted in
little-endian order:

\begin{figure}[H]
  \centering
  \begin{bytefield}{32}
    \bitheader[endianness=little]{0-31} \\
    \bitboxes*{1}{00000001} &
    \bitboxes*{1}{00000000} &
    \bitboxes*{1}{10000010} &
    \bitboxes*{1}{11100000} &
  \end{bytefield}
\end{figure}

Finally, we convert to hexadecimal:


\begin{figure}[H]
  \centering
  \begin{bytefield}{32}
    \bitheader[endianness=little]{0-31} \\
    \bitbox{8}{01} &
    \bitbox{8}{00} &
    \bitbox{8}{82} &
    \bitbox{8}{e0} &
  \end{bytefield}
\end{figure}

So the hexadecimal encoding of this instruction is {\tt 010082e0}. We can put
that into a file, let's call it {\tt add.txt}. The {\tt xxd} command can be
used to create a binary from it:

\begin{verbatim}
$ xxd -p -r add.txt add.bin
\end{verbatim}

And then we can run the file using the provided runner, which lets us give
arguments as command-line arguments, puts them into the designated registers,
and runs our binary code:

\begin{verbatim}
cs30f@pi-cluster-013:~/pa1/$ pa1-runner add.bin 3 -7
f(3, -7):
  r0 = fffffffc (-4)
  r1 = 3        (3)
  r2 = fffffff9 (-7)
  r3 = 7efc2aec (2130455276)
  r4 = 0        (0)
  r5 = 0        (0)
  r6 = 110f4    (69876)
\end{verbatim}

This command \emph{must} be run on {\tt pi-cluster}. Note that the values in
\r{3} to \r{6} aren't set by the sample adding program, so you might see
different values show up there. But \r{0} has the sum, \r{1} has the first
input, and \r{2} has the second input.

A few extra notes:

\begin{itemize}
\item We can use {\tt xxd} to see our instruction in binary again:

\begin{verbatim}
$ xxd -b add.bin
00000001 00000000 10000010 11100000
\end{verbatim}

Or in hexadecimal:

\begin{verbatim}
$ xxd -p add.bin
010082e0
\end{verbatim}

\item An abbreviated process for getting to the appropriate hexadecimal from
the conventional binary notation is to convert to hexadecimal, and then simply
reverse the order of the four bytes, so the first byte becomes the last, the
last the first, and the middle two are swapped:

\begin{figure}[H]
  \centering
  \begin{bytefield}{32}
    \bitheader[endianness=big]{0-31} \\
    \bitboxes*{1}{11100000} &
    \bitboxes*{1}{10000010} &
    \bitboxes*{1}{00000000} &
    \bitboxes*{1}{00000001} &
  \end{bytefield}
\end{figure}

\begin{figure}[H]
  \centering
  \begin{bytefield}{32}
    \bitheader[endianness=big]{0-31} \\
    \bitbox{8}{e0} &
    \bitbox{8}{82} &
    \bitbox{8}{00} &
    \bitbox{8}{01} &
  \end{bytefield}
\end{figure}

\begin{figure}[H]
  \centering
  \begin{bytefield}{32}
    \bitheader[endianness=little]{0-31} \\
    \bitbox{8}{01} &
    \bitbox{8}{00} &
    \bitbox{8}{82} &
    \bitbox{8}{e0} &
  \end{bytefield}
\end{figure}
\end{itemize}


\section{Programs to Write}

You will fill in the provided text files, each with a particular formula using
the instructions above. You will hand in your plain-text files with
instructions in hexadecimal, using the process described above. In all cases,
assume that initially $x$ is stored in \r{1}, $y$ is stored in \r{2}, and the
result should appear in \r{0}. You may use any registers from \r{0} to \r{6}
for your computation.

\begin{enumerate}
\item In {\tt sub2.txt}, compute $x - y - y$
\item In {\tt submul.txt}, compute $(x - y) * y$
\item In {\tt addsubmul.txt}, compute $(x - y) * (x + y)$
\item In {\tt quad.txt}, compute $2x^2 + yx - 7$
\end{enumerate}


\section{README}

Along with your code, please also submit a README file which contains your 
answers to the following questions:

\begin{enumerate} \item There are three mystery files that contain hexadecimal
instructions in your checkout. Your job is to figure out and explain what these
mystery binaries do (they are short). You might do so via testing, or by
converting the hexadecimal instructions back to binary code, and checking for
opcodes. You will probably want to use commands like {\tt xxd -p -c 4
mystery1.bin} to do this. In your explanation, you should explain how you get
your final answers. For instance, you can say: ``the opcode of the first
instruction is 0000, which represents a AND operation.'' or ``$R_d$ of the
second instruction is 0001, which stores the result to register 1'' 

Also, provide at three examples for each mystery file with different values of
x and y that you pass as arguments to run the program. After you figure out
what operations are used in these files, explain the meanings of these
operations and results. (Notice that the arguments you passed in and the answer
are all in decimal values. It might be helpful to convert them to other bases
in order to understand the meaning of these operations)

\item Starting from the instruction format in the textbook and this writeup, 
we have to do two apparent reversing operations: one on the whole instruction, 
and one on each byte. Please explain why we need these two reverses in your
own words.
\item Consider the machine code corresponding to r0 = r1 + r2. If we wanted to 
change the instruction to use some other register instead of r1, how would the 
machine code change and why?
\end{enumerate}

Please limit your answer to each question to no more than 100 words and present
them as clearly as possible. For the mystery files, use 100 words or less for
each file.

\section{Handin}

Commit and push your four text files to the Github repository that was created
for you by 11:59PM on Tuesday, October 10. You can push up to one day late for
a 20\% penalty. After you push, make sure to check on Github that the files are
actually there; we will mark all of the repositories for grading a few minutes
after midnight and grade precisely what is there.

Your handin should include:

\begin{itemize}
\item Four .txt files containing hex values ({\tt addsubmul.txt}, {\tt
sub2.txt}, {\tt submul.txt}, and {\tt quad.txt}) that are instructions to
calculate the formulae above.
\item A single README.txt file that has answers to the open-ended questions
above, as specified.
\end{itemize}

\end{document}
%%% Local Variables:
%%% mode: latex
%%% TeX-master: "description"
%%% End:

